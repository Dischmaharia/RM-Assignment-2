\documentclass{article}
\usepackage[utf8]{inputenc}

\title{RM Assignment 2}
\date{February 2019}

\begin{document}

\maketitle

\begin{tabular}{|l|l|l|}
\hline
     \textbf{Student name} & \textbf{P. No.} & \textbf{Contribution} \\
     \hline
     Sylvain Roncoroni & 990109T018 & \\
     Claudio Bertozzi & 930206T578 & \\
     Hassan Mehmood & 9205054837 & \\
     Ehtasham ul haq & 9111168770\\
     \hline
\end{tabular}


\section{Scenario 1}
\subsection{Assumptions}
We assume that the employee who participate in the experiment know both methods.

\subsection{Research questions}
We are assuming that TDD leads to a better quality in fewer time. For this reason we want to ensure these two hypothesis.
H1: TDD improves the quality of the software compared to TLD.

\subsection{Research methodology}
For the first scenario we suggest to do an experiment in which we ask half of the people participating to write code using TDD and the other half using TLD. Then using criteria/metrics to determine the quality of the code they wrote (reliability, number of defects, security, etc.), we can see which method gives the best results.\\
Independent variables:
\begin{itemize}
    \item
\end{itemize}
Dependent variables:
\begin{itemize}
    \item
\end{itemize}

\subsection{Justification}
---??? An experiment is required because we want to measure the benefits and drawbacks of each method.\\
An other option would be to do a case study in which a lot of different project would be compared. But it would be extremely difficult to isolate the the dependent variables to justify the findings.


\section{Scenario 2}
\subsection{Assumptions}
We assume that the employee who answer the survey have already worked with both TDD and TLD.

\subsection{Research questions}
Q1: Do the employees from the company X prefer to work with TDD compared to TLD?

\subsection{Research methodology}
Create a questionnaire to know which method is favored by the developers of the company but also if they are ready to change and adapt to the other method and if they have any reservations regarding the new method.
Population is the employees; sampling strategy is to simply ask all employees to answer the survey. \\
Independent variables:
\begin{itemize}
    \item Employees
\end{itemize}
Dependent variables:
\begin{itemize}
    \item Preferred method measured by percentage of employee who prefers it
    \item Percentage of developers ready to change for the other method
    \item Reservations of employees regarding the new method
\end{itemize}

\subsection{Justification}
An experiment could also take place, so that the employees first had to work with the different techniques. However, in this scenario, the only aim of the study is to know which method is preferred by the employees. So, the easiest way to know is to simply ask for their opinion and the survey is well-fitted for this task. 


\section{Scenario 3}
\subsection{Assumptions}

\subsection{Research questions}
Q1: How can the current used agile development process in the company X be improved to become more efficient?

\subsection{Research methodology}
In this scenario we would do a case study during which we analyze the performance of the agile development process. We are interested in the following points:
\begin{itemize}
    \item The current process – global view on how they work
    \item Possible improvements
    \item The problems the employee face with the current process – problems linked to the way they work, not the work itself
\end{itemize}
To do that, we can observe their agile development process during multiple projects (if they are small enough) or multiple sprints. Then interview some developers to help us get the global view of the way they work and ask their opinion about the problems they may see with this method of working and how this can be improved/problems be solved. Additionally, we will search for other research papers which provide solutions for the problems found.

\subsection{Justification}
The only choice is a case study because the goal is to benchmark the performance of a specific process in a company which first has to be investigated.


\section{Scenario 4}
\subsection{Assumptions}

\subsection{Research questions}
H1: Algorithm A is more effective than algorithm B. \\
H2: Algorithm A is more efficient than algorithm B.

\subsection{Research methodology}
For this scenario we suggest to do en experiment. First we have to create a source code with corresponding test cases. We will need test cases that will be detected by the algorithms or else it is not that useful. Run both algorithms on the test case suite and measure the following:
\begin{itemize}
    \item Time taken to execute
    \item Number of test cases that fails and are detected by the algorithm (true positive)
    \item Number of test cases that succeeds and are not detected (true negative)
    \item Number of test cases that fails and are not detected (false negative)
    \item Number of test cases that succeeds and are detected (false positive)
\end{itemize}
Analyze the results and compare the time taken and the precision of each algorithm for each test case suite and finally make a global comparison.

\subsection{Justification}
We want to measure performance of two algorithms, so we need to make an experiment to get values on which it’s possible to work and make an analysis to compare the algorithms.
The time taken to execute is used to compare the effectiveness.
The other measures are used to calculate the precision and the recall of each algorithm which give us an idea of the relevance of the results given.

\end{document}
